% !TEX root =../main.tex

\chapter{Fachlicher Entwurf des WebShops (3 Seiten)}

In den obigen Kapiteln vergleicht der Autor die Dienstleistungen, die von fünf Online-Shopping-Stores in der Benutzer-Community angeboten werden, Echtzeitkommunikation, Zahlungsmethoden und Auswertung. Unter ihnen belegte Alibaba mit 54 Punkten den ersten Platz, gefolgt von Amazon und Ebay auf den Plätzen zwei und drei. BANDCAMP und Zalando erzielen die niedrigste Punktzahl.

Im Vergleich dazu können wir sehen, dass die drei besten Ergebnisse diejenigen mit der größten Anzahl an Kunden sind und bei den Nutzern sehr beliebt sind. Unter ihnen bietet Alibaba Mehrwertdienste als andere Unternehmen.

Mehrwertdienste sind sehr wichtig für die Entwicklung des Unternehmens: Ihr On\-line-Shop ist nicht länger nur ein Geschäft, und seine Gewinnquellen sind tendenziell diversifiziert. Zum Beispiel, in Bezug auf Zahlungsmethoden, haben Amazon und Alibaba ihre eigenen Finanzderivate entwickelt.Amazon und Banken kooperieren, um Kredit-Shopping bieten, so dass Benutzer nicht mehr durch ihre Zahlungsfähigkeit, Amazon Umsatz und Zinsen aufgrund von Kreditaufnahme eingeschränkt werden können Erhöhen Sie den Unternehmensumsatz. Alibaba hat ein eigenes Zahlungssystem entwickelt, um \glqq{}papierlose\grqq{} Zahlungen zu ermöglichen. Gleichzeitig, weil in Alipay die Zins- und Kapitalerträge, die durch die riesige Währungseinlage generiert werden, nicht geschätzt werden können.

Die Nutzergemeinschaft, Produkte und Evaluierungsservices sorgen für ein besseres Einkaufserlebnis für die Nutzer, sodass Nutzer länger im Internet bleiben können. Der Fachbegriff "Benutzerviskosität"(Englisch: Customer stickiness) wird verwendet, um zu beschreiben, dass eine erfolgreiche Website die Nutzer dazu anlockt, lange Zeit zu bleiben und somit einen kontinuierlichen Verkauf und Ausbau zu erreichen. Je höher die Benutzerviskosität des Nutzers ist, desto beliebter ist die Website und desto höher ist ihr Geschäftswert. Ein typisches Beispiel ist YOUTUBE, und Anzeigen werden eher auf Websites platziert, die bei Nutzern beliebt sind und bereit sind, für längere Zeit online zu bleiben.

Customer Stickness ist für die moderne Geschäft spielt eine große Rolle. Laut \textcite{bradlow-wharton} Professor der Wharton University of Pensylvania ist sie wie folgt definiert: \glqq{}Customer stickiness is the increased chance to utilize the same product or service that was bought in the last time period. Due to Levi’s customer stickiness for its jeans, people who try Levi’s, buy pairs over and over again.\grqq{}

Die Websites mit geringer Benutzerviskosität zeigen sich darin, dass sich der Benutzer versehentlich durch die Suche erinnert oder sich anmeldet. Nachdem die Nutzung abgeschlossen ist, wird der Benutzer nicht mehr erinnert, der bereitgestellte Dienst wird normalisiert und die \glqq{}überzählige\grqq{} Erfahrung des Benutzers ist nicht zufriedenstellend. Die Website überlebt hart.\\
\parencite{localytics}

In der kooperativen Einkaufswebsite, zu der sich der Autor verpflichtet fühlt, zielt es daher auch darauf ab, mehr Benutzer anzuziehen, mehr Dienste und Funktionen bereitzustellen, das erwartete Einkaufserlebnis des Benutzers zu befriedigen und die Viskosität des Benutzers zu verbessern.

Unsere Website nutzt die Vorteile einer beliebten Website und hat mehr Funktionen entwickelt, die nicht nur die Mängel anderer Websites ausgleichen, sondern auch die Interaktion zwischen den Nutzern erheblich erhöhen, was das Einkaufen komfortabler und interessanter macht.

% TODO: Kann das nachfolgende gelöscht werden?

\section{Notizen}

Eher Social E-Commerce oder?

Der Begriff Social Commerce wurde dem gleichnamigen Buch \parencite{turban:sc} entnommen. Den Begriff würde ich nur ungern abwandeln. 

Welche [Geschäftsmodelle] gibt es in der Richtung überhaupt? Kläre mich auf. 

\begin{itemize}
\item Free (z. B. Facebook)
\item Freemium (z. B. Spotify, XING, GitHub)
\item Long Tail – Angebot von Nischenprodukten (z. B. Itunes)
\item Marktplatzmodell (z. B. Amazon, Ebay)
\end{itemize}

Es gibt unterschiedliche Geschäftsmodelle, für meine Arbeit werden Marktplatzmodell benutzen, die Plattform für die Käufer kostenlos, für die Verkäufer wird Freemium angewendet.

Wie könnte dieses Konzept aussehen? 

Konzept würde aus folgenden Teilen bestehen: 
\begin{itemize}
\item Geschäftsmodell 
\item Oberflächendesign 
\item Funktionaltät/Features 
\item Umsetzung von Anwendungsfällen beschreiben 
\end{itemize}

Welche Kommunikationskanäle sollen genutzt werden? 

Zwischen Kunden

\begin{itemize}
\item Text (Instant Messenging) 
\item Sprache 
\item Video 
\item Bewertungen 
\end{itemize}

Zwischen Kunde und verkäufer 

\begin{itemize}
\item Text 
\item Sprache 
\item Bewertungen 
\end{itemize}

Welche Methode nutzt du zur Untersuchung? Wie soll diese Erfolgen?

\begin{itemize}
\item Case-based Evidence als Methode zur Untersuchung 
\item Auf Basis von Anforderungen wird überprüft, ob eine bereits existierende Software diese Anforderung erfüllt 
\item Beispielsweise könnte eine Anforderung sein, dass der Nutzer seinen Avatar in einer 3D-Umgebung bewegen und steuern muss 
\item Computer spiele haben genau diese Anforderung bereits erfolgreich erfüllt 
\end{itemize}

Was ist deine Hypothese bzw. welches Ergebnis erwartest du?

\begin{itemize}
\item Durch die Verbindung von E-Commerce, Social Media und Computerspielen kann kooperatives Einkaufen ermöglicht werden 
\item Kooperatives Einkaufen eignet sich besonders für subjektive Kaufentscheidungen 
\item Shops liefern bisher eher Hilfe bei objektiven Kaufentscheidungen (Testberichte und Bewertungen fremder Nutzer) 
\item Für subjektive Einschätzungen ist vor allem die Meinung von Freunden wichtig 
\end{itemize}

Welche Anwendungsfälle – die musst du schon hier darlegen.

\begin{itemize}
\item Zwei Personen verabreden sich zum gemeinsamen Einkaufen 
\item Während des Einkaufens benötigt eine Person Hilfe bei einer Kaufentscheidung und lädt einen Freund ein ihr zu helfen 
\item Kunde kann mit Verkäufer kooperativ interagieren 
\item Kunde kann Geld zu anderem Kunden transferieren
\end{itemize}
