% !TEX root =../main.tex

\chapter{Fachlicher Entwurf des WebShops (3 Seiten)}

Eher Social E-Commerce oder?

Der Begriff Social Commerce wurde dem gleichnamigen Buch von Efraim Turban, Judy Strauss und Linda Lai aus dem Jahr 2015 entnommen. Den Begriff würde ich nur ungern abwandeln. 

Welche [Geschäftsmodelle] gibt es in der Richtung überhaupt? Kläre mich auf. 

\begin{itemize}
\item Free (z. B. Facebook)
\item Freemium (z. B. Spotify, XING, GitHub)
\item Long Tail – Angebot von Nischenprodukten (z. B. Itunes)
\item Marktplatzmodell (z. B. Amazon, Ebay)
\end{itemize}

Es gibt unterschiedliche Geschäftsmodelle, für meine Arbeit werden Marktplatzmodell benutzen, die Plattform für die Käufer kostenlos, für die Verkäufer wird Freemium angewendet.

Wie könnte dieses Konzept aussehen? 

Konzept würde aus folgenden Teilen bestehen: 
\begin{itemize}
\item Geschäftsmodell 
\item Oberflächendesign 
\item Funktionaltät/Features 
\item Umsetzung von Anwendungsfällen beschreiben 
\end{itemize}

Welche Kommunikationskanäle sollen genutzt werden? 

Zwischen Kunden

\begin{itemize}
\item Text (Instant Messenging) 
\item Sprache 
\item Video 
\item Bewertungen 
\end{itemize}

Zwischen Kunde und verkäufer 

\begin{itemize}
\item Text 
\item Sprache 
\item Bewertungen 
\end{itemize}

Welche Methode nutzt du zur Untersuchung? Wie soll diese Erfolgen?

\begin{itemize}
\item Case-based Evidence als Methode zur Untersuchung 
\item Auf Basis von Anforderungen wird überprüft, ob eine bereits existierende Software diese Anforderung erfüllt 
\item Beispielsweise könnte eine Anforderung sein, dass der Nutzer seinen Avatar in einer 3D-Umgebung bewegen und steuern muss 
\item Computer spiele haben genau diese Anforderung bereits erfolgreich erfüllt 
\end{itemize}

Was ist deine Hypothese bzw. welches Ergebnis erwartest du?

\begin{itemize}
\item Durch die Verbindung von E-Commerce, Social Media und Computerspielen kann kooperatives Einkaufen ermöglicht werden 
\item Kooperatives Einkaufen eignet sich besonders für subjektive Kaufentscheidungen 
\item Shops liefern bisher eher Hilfe bei objektiven Kaufentscheidungen (Testberichte und Bewertungen fremder Nutzer) 
\item Für subjektive Einschätzungen ist vor allem die Meinung von Freunden wichtig 
\end{itemize}

Welche Anwendungsfälle – die musst du schon hier darlegen.

\begin{itemize}
\item Zwei Personen verabreden sich zum gemeinsamen Einkaufen 
\item Während des Einkaufens benötigt eine Person Hilfe bei einer Kaufentscheidung und lädt einen Freund ein ihr zu helfen 
\item Kunde kann mit Verkäufer kooperativ interagieren 
\item Kunde kann Geld zu anderem Kunden transferieren
\end{itemize}
