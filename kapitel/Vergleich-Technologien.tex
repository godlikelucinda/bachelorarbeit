% !TEX root =../main.tex

\chapter{Vergleich möglicher Technologien (5 Seiten)}


\section{die Entwicklung von der Webseite mittels Webgl und three.js}

WebGL (Web Graphics Library) ist eine Javascript-API zum Rendern interaktiver 3D und 2D Grafiken mittels eines kompatiblen Web-Browsers ohne Einsatz zusätzlicher Plugins. Mit WebGL steht eine API zur Verfügung, die an OpenGL ES 2.0 angelehnt ist und deren Inhalte mittels eines <canvas> Elements dargestellt werden

Schwerpunkt: Performance


\section{zusätzliche Plugins auf die Browser installieren und Cache lokal speichern}

um schnellstmöglich die riesig große 3D- Daten auf die Webseite zu zeigen, wir können die Nutzer Plugins auf ihre Browser installieren lassen. Dann werden die besuchte Daten auf den lokalen Rechner gespeichert, der Server muss nicht bei jedem Besuch wiederholt downloaden und nur regelmäßige Aktualisierung (update) machen und entlasten 

Schwerpunkt: ob die Nutzer die Installierung akzeptieren wollen


\section{auf C Sprache basierende Frameworks}

Vorteile : beste Performance, schöne Szenen

sowie DOTA World of Warcraft  bzw. WoW solche groß Netzwerk Games die Nutzer sollen die Softwarepaket herunterladen
 und Dateien extrahieren dann executable file ausgeführt werden. Wir verwenden hauptsächlich C\# / C++ Assembler 
 -Sprache Shader-Sprache Skriptsprache und C oder Java.

Außerdem brauchen wir auch ein Motor. Er bestimmt die grundlegenden Operationen und Effekte einer 3D- Webseite. Die allgemeine Entwicklungsarchitektur ist von unten nach oben ist  DirectX (Windows-Plattform) – Motor- kooperatives Einkaufen. Direct X ist basis auf OpenGL und  stellt eine einheitliche Schnittstelle zu allen Grafikkarten dar, um ein Programmiermodell mit Hardware-Rendering anzubieten. die Direct X-Schnittstelle hat gute Performance aber  nur einfache Malen-Funktionen.  Es ist notwendig, wir soll  durch Auswahlen und Encapsulating  die abstraktere  Entwicklungsschnittstellen und Framework  realisieren , die als eine Grafik-Engine in der Spiel-Engine verstanden werden können. Diese  Teile der Entwicklung ist C / C ++ und wenige Assembler. Der Motor ist kompliziert und integriert, inklusive Grafik-Engine, Audio-Engine, Kunstressourcenverwaltung, Script-Engine usw. Bei  Assembler verbessert die Effizienz und bei Script-Sprache muss man nicht etwas wiederholt tun.
