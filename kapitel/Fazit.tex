% !TEX root =../main.tex

\chapter{Fazit}

Diese Bachelorarbeit wurde mit der Zielsetzung durchgeführt, Kooperatives Einkaufen zu ermöglichen und die Mängel anderer Webshops zu vermeiden. Mit Hilfe des kooperativen Einkaufens kann man ein besseres Nutzererlebnis erzielt werden.

Es wurde festgelegt, die Konzepte von Social Commerce zu nutzen, um die Interaktion zwischen Nutzern zu verstärken. Dadurch sollte erreicht werden, die potentielle Bedürfnisse von Beratung und Feedback besser zu erfüllen.

Im Rahmen dieser Arbeit wurde untersucht, welche Stärken und Schwächen heutige populäre Webshops besitzen und welche Unterschiede sich bei einem Vergleich offenbaren. Detailliert betrachtet wurden hierbei Alibaba, Amazon und Ebay. Bei der Untersuchung dieser Webshops wurden Vorteile erkannt, die für das Szenario des kooperativen Einkaufens übernommen werden können. Aufgrund einer besseren Werkzeugunterstützung wurde entschieden, BPMN und Usecase-Diagramme am Ende für die Einkaufsprozesse zu nutzen. Für das kooperative Einkaufen bestand das Ziel, dass die User Stickiness erhöht und mehrere Gewinnsquellen erzielt werden können.

Die Kernfunktionen konnten im Rahmen dieser Bachelorarbeit erfolgreich für Kunden in der Webshops konzipiert werden. In der Nutzer-Community steht nicht nur Hilfe und Richtlinie, sondern auch ein Platz für Share, Like, Bewertung und Wirkung des Produkts mitzuteilen. Als Zahlungsmethode sollten nicht lediglich Kreditkarte oder Girokarte zum Einsatz kommen, sondern ebenfalls Funktionen wie \term{pay by others}. Durch die Verwendung von Echtzeitkommunikation und Freundeslisten kann die User Stickiness und die Qualität des Service weiter erhöht werden.
