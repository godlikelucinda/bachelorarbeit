% !TEX root =../main.tex

\chapter{Theoretischer Hintergrund und Konzepte}


\section{Definition vom Social Network}

Nach Charu C Social Network Data Analytics  (Seite 2 )„ in general, a social network is defined as a network of interactions or relationships, where the nodes onsist of actors, and the edges consist of the relationships or interactions between these actors.“ Charu C meint, „the concept of social network is not restricted to the specific case of an internet- based social network such as Facebook, such interactions may be in any conventional or non-conventional form, whether they be face to face interactions, telecommunication interactions, email interactions or postal mail interactions“


\section{Welche Konzepte von Social Netwerk können angewendet werden}

Soziale Netzwerke existierten , um die Bedürfnissen bestimmter Personen zu erfüllen. Die Klassifikationen nach unterschiedliche Funktionen:

basis auf human Interaction sind Facebook, Myspace, online Lebenslauf LinkedIn, Paarsuche Web : Parship usw;  Basis auf Blog-Publishing- und User-Attention-Services, diese Beziehungen zwischen die Nutzer gestalten das Netzwerk, z.B Twitter, Follow 5. 

basis auf sharing online media Content Flickr, Youtube oder Delicious.
Basis auf real time Kommunikation, z.B Whatsapp, skype...

\begin{itemize}
\item real time Kommunikation gibt es viele Maßnahmen, um die Nachricht zu schicken, z.B. Text (Instant Messenging) , Sprache , Video , die Vorteile ist offensichtbar,  Senden und Erhalten von der Nachricht sind rapid
\item die Akteure vom Facebook können unterschiedlich sein, z.B. Persönlicher Account, persönliche Homepage, Business-Homepage, kommerzieller Werbekonto, Business-Management-Plattform, Event, Aktivität, Veranstaltung, News, usw.
\item Expert Discovery in Networks ( Chara C, Social network data analytics, Seite 11) „ Social networks can be used as a tool in order to identify experts for a particular task. given the activities of candidates within a context, we first descaribe methods for ecaluating the level of expertise for each of them. Many complex tasks often require the collective expertise of more than one expert“
\item „social Tagging:  much of the interaction between users and social networks occurs in the form of tagging, in which users attach short descriptions to different objects in the social network, such as images, text, video or other multimedia data.
\item Randoms walks and their Application in Social Networks. Ranking is one of the most well Kown methods in web search.Ranking Users in Social Networks with Higher-Order Structures:
\end{itemize}