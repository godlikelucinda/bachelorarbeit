% !TEX root =../main.tex

\chapter{Einleitung}


\section{Anwendungsfall}

Es passiert oft, man ist beim online Einkaufen unschlüssig, ein Produkt auszuwählen und eine Transaktion erledigen. Die Gründe sind folgend:

\begin{itemize}
\item Vorschlagen erhalten wollen
\end{itemize}

Falls die Alternative vorhanden sind, Käufer können nicht entscheiden, davon auszuwählen.

\begin{itemize}
\item fehlende Erkenntnisse\\
Nach die Forschung von Akerlof, G. (1970). The market for lemons :“ Asymmetrische Information, bezeichnet den Zustand, in dem zwei Vertragsparteien bei Abschluss und/oder Erfüllung eines Vertrags oder Marktteilnehmer nicht über dieselben Information verfügen. Die Auseinandersetzung mit Problemen, die aus asymmetrischen Informationen resultieren“
\end{itemize}

z.B. beim Kaufen von einem Auto, nicht jeder Käufer besitzt gute Erkenntnisse über die Ausstattungsdetails, Motor, Getriebe usw. Aber finden die Verkäufer, alle Autos, die sie verkaufen, sind prima. In diesem Fall braucht der Käufer die Hilfe von einem Freund, der relativ solche gute Erkenntnisse hat.

Deshalb dieses Kooperative Einkaufen geeinigt besonders für die wertvolle, spezielle und fachliche Produkte, Computer, Haus, Kamera, usw.

\begin{itemize}
\item Fehlende Erfahrung\\
Manche Produkte oder Dienstleistungen, muss man nicht Erkenntnisse haben. z.B. Reise buchen, man braucht mehrere Informationen und Tipps der Urlaubsorte. Die Erfahrungen und Bewertungen von anderen gegangenen Leute sind nötig.
\item Finanzierung, (pay by others)\\
wenn ein Nutzer nicht für die ausgewählte Produkte bezahlen kann, er will eine finanzielle Überstützung von einem anderen Bekannte bekommen.
\item Die andere Falls ergänzen später
\end{itemize}


\section{Motivation}

Soziale Netzwerke erfreuen sich großer Beliebtheit und ihre Betreiber zählen zu den größten Internetunternehmen der Welt. Laut Sokolov (2017) kann das soziale Netzwerk Facebook bereits 2 Milliarden Nutzer vorweisen. In einer durch NASDAQ 100 (2018) aufgestellten Statistik belegt Facebook den 5. Platz der größten Internetunternehmen, gemessen an ihrem Börsenwert. Mittermeier (2016) listet Facebook gar auf dem dritten Platz. Es ist anzumerken, dass in beiden Statistiken Facebook direkt hinter dem Online-Versandhändler Amazon gelistet wird. Am Börsenwert gemessen
scheint der Online-Handel die sozialen Netzwerke sogar noch zu überflügeln. Es scheint zumindest sicher, dass sowohl soziale Netzwerke als auch Web-Shops sich eines großen Erfolgs erfreuen.


\section{Ziele und Hypothesen}

Kooperatives Social Commerce
- Wie können Web-Shops mit Sozialen Netzwerken kombiniert werden?

Im Rahmen dieser Arbeit sollen die Konzepte und Mechanismen der meistgenutzten Web-Shops mit einer Auswahl an Social-Media-Angeboten untersucht und verglichen werden. Es soll untersucht werden, welche unterschiedlichen Geschäftsmodelle zum Einsatz kommen und wie diese kombiniert werden können. Abschließend soll ein Konzept eines Web-Shops entwickelt werden, der Konzepte von Social-Media-Webseiten mit denen eines Web-Shops vereint. Dieses Konzept soll Möglichkeiten aufzeigen, wie in einem Web-Shop die Tätigkeit des Einkaufens mit anderen Nutzern geteilt und kooperativ erlebt werden kann.

Die Kooperation soll hierbei komplett in der konzeptionierten Webapplikation abgebildet werden. Es sollen keine Zusätzlichen Kommunikationskanäle notwendig sein.

Innerhalb dieser Bachelorarbeit soll darüber hinaus die These untersucht werden, das Kooperation in Webshops im Stil von Online Computerspielen abgebildet kann.


\section{Methoden}

Die Bachelorarbeit beginnt damit, die Konzepte von E-Commerce, Social Media und Social Commerce zu betrachten. Anschließend wird untersucht wie im Bereich des Online Gamings eine Kooperation von Nutzern ermöglicht und gefördert wird. Darauf folgend werden Anwendungsfälle definiert, welche ein Webshop unterstützen muss, um kooperatives Einkaufen zu ermöglichen. Auf der Basis dieser Anwendungsfälle wird eine Untermenge der Untersuchten Konzepte ausgewählt und ein Entwurf eines Prototyps angefertigt.

Die Bachelorarbeit schließt mit der Implementierung des Prototypen ab, um nachzuweisen, dass der angefertigte Entwurf tatsächlich realisierbar ist. Es wird eine Bewertung der implementierten Lösung vorgenommen, offene Punkte werden dargelegt und es wird ein Fazit gezogen.


\section{Artefakte}

bei der mit Web 3D gestalteten Webseite, kann die Nutzer kooperativ einkaufen, das Bedarf von den Vorschlägen, Einkaufen direkt auf dieselbe Projekt erfüllen, muss nicht mehr mittels andere Applikation vornehmen.


\section{Abgrenzung, etc.}

In der Arbeit beobachten wir die technische Probleme nicht, z.B. wie die Performance lauft, nach die Webseite mit WEBGL gestaltet wird. wie Daten gespeichert werden? Cloud oder ein verteiltes Speichersystem? Wie kann man sicherstellen, die Server gut zu laufen und die Kaufprozess fließend durchzuführen? wenn die Nutzer und Daten immer zu nehmen? Unabhängige Server von Transaktion, Daten und Dateien? Ist Caching nötig zu verwenden ?. wie die Zugriffsgeschwindigkeit zu erhöhen, CDN und Reverseproxy ?
