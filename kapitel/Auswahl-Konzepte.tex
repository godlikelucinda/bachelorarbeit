% !TEX root =../main.tex

\chapter{Auswahl an Konzepten Festlegen}


\section{Voraussetzung bei der Entwicklung}


% TODO: Was bedeutet die Viskosität zwichen Nutzer und Webseite?
\subsection{Viskosität zwischen Nutzer und die Webseite verstärken}

% TODO: Vollständigen Namen des Institus Nielsen angeben. Vielleicht als Fußnote.
% TODO: Genaues Datum angeben bei `Nielsens am Dienstag veröffentlichte ...'
% TODO: Bitte wörtliches Zitat auf Korrektheit überprüfen: `ist drei Viertel der Online-Nutzer auf der Welt zugreifen können.'
% TODO: Genauen Zeitraum angeben bei `haben im vergangenen Monat ...'. Besser ist `haben im Mail 2018 ...'.

Laut dem Marktforschungsunternehmen Nielsen verbringen Nutzer mit Instant Messaging, dem Schreiben von Kommentaren, Bloggen, Teilen und \glqq{}Liken\grqq{} durchschnittlich 22\% ihrer Online-Zeit. Nielsens am Dienstag veröffentlichte Statistik zeigte, dass Menschen alle 4,5 Minuten ihrer Online-Zeit eine Minute in sozialen Netzwerken verbringen. Internetnutzer verbringen somit jeden Monat 110 Milliarden Minuten in sozialen Netzwerken und Blogs. Die Umfrage sagt, dass dies dass erste Mal ist, dass ein soziales Netzwerk oder ein Blog \glqq{}ist drei Viertel der Online-Nutzer auf der Welt zugreifen können.\grqq{} Das ist eine Steigerung von 24\% gegenüber dem gleichen Zeitraum des Vorjahres. Die beliebtesten Social-Networking-Marken sind Facebook und YouTube. Online-Nutzer haben im vergangenen Monat 13 Milliarden Videos auf YouTube gesehen. Facebook meldete, dass seine Nutzer 2 Milliarden Videos pro Monat sehen.

% TODO: (Nielsen-Studie: Facebook und Google haben erfolgreichste Apps Jonas Wagner) in litertur.bib einfügen. `

Laut Nielsens Studie hat Facebook in den globalen Online-Stunden das Rampenlicht übernommen. Fast 500 Millionen Nutzer verbringen jeden Monat sechs Stunden damit.
Was die Netzabdeckung betrifft, übernimmt Google die Führung. Laut Nielsen-Statistiken besuchen 82\% der Internetnutzer weltweit jeden Monat diese Website, und die durchschnittliche Suchzeit beträgt 1 Minute und 20 Sekunden.(Nielsen-Studie: Facebook und Google haben erfolgreichste Apps Jonas Wagner)

D.h. die Länge der Zeit, welche ein Nutzer auf einer Webseite verbringt, ist ein Zeichen dafür, ob diese Webseite beliebt ist und ob die Viskosität stark ist.

Ein wichtige Frage bei dem Entwurf einer Webseite ist, wie man Nutzer motiviert, mehr Zeit auf dieser Webseite zu verbringen. Durch interessante Funktionen und Veranstaltungen?


\subsection{Online Shops welche mehr Produkte verkaufen sind beliebter}

Nach der Statistik von Alexa in Abbildung \vref{fig:alexa}, sind die Top 10 Webshops mit dem höchsten Umsatz unter anderem Amazon, Netflix, Ebay, Walmart, Etsy, Steam und Ikea. Sie alle haben eine Gesamtheit: Es ist eine hohe Vielfalt an Produkten verfügbar.

\begin{figure}
	\centering
	\includegraphics[width=1\textwidth]{bilder/alexa.png}
	\caption{Alexa}
	\label{fig:alexa}
\end{figure}


\section{Freunde organisieren}

Man verwendet Listen, um  Freunde organisieren. Mit einer Liste kann man die Meldungen in seinem News-Feed filtern oder Aktualisierungen mit bestimmten Personen teilen. Z. B. mit Arbeitskollegen oder Freunden, die in der Nähe wohnen. Man kann jederzeit Freunde hinzufügen order entfernen. \parencite{facebook:help}

\textbf{Enge Freunde:} Freunde, mit denen man möglicherweise alles teilen möchte.

\textbf{Bekannte:} Personen, mit denen man eventuell weniger teilen möchte.

\textbf{Eingeschränkt:} Diese Liste ist für Personen, die man als Freunde hinzugefügt hat, mit denen man aber nichts teilen möchte. Z. B. einem Arbeitskollegen oder Vorgesetzten.

Man kann außerdem selbst definierte Listen erstellen,  um selbst Freunde in Gruppen zusammenzufassen. Der Nutzer legst fest, wer zu einer bestimmten Liste hinzugefügt wird und welche Privatsphäre-Einstellungen (gegebenenfalls) gelten. Es ist zu beachten, dass Freunde nicht benachrichtigt werden, wenn sie zu benutzerdefinierten Liste hinzufügt werden.


\section{Echtzeitkommunikation mit anderen Nutzern, Verkäufern und dem Kundenservice der Plattform}

Ein potentieller Nachteil von Amazon ist, dass es schwer ist mit dem Verkäufer oder dem Kundenservice persönlich in Kontakt zu treten. Dadurch kann das Kommunizieren und Lösen von Problemen zeitaufwändig und kompliziert ausfallen.

Durch die Verwendung von Echtzeit-Kommunikation, können Probleme unter Umständen deutlich schneller und unkomplizierter gelöst werden. Dies könnte das Einkaufserlebnis deutlich verbessern.


\section{Offline Kontakte}

Gleichzeit eine URL als Nachricht per E-Mail und SMS versenden, welche die Produktinformation beinhaltet. Freunde oder Bekannte müssen sich so nicht noch einmal anmelden, sonder können direkt die URL anklicken. Dadurch werden sie automatisch in die Bestehende Sitzung des Käufern eingeloggt und können sofort Vorschläge unterbreiten und bei der Entscheidungsfindung unterstützen.


\section{Nach dem Handel, Erfahrung und Bewertung teilen und Produkt mit Freunden teilen}


\section{Pay by others}

Wenn ein Kind die finanzielle Hilfe der Eltern benötigt, können sich die Eltern auch in dem genutzten Webshop einloggen und entscheiden, ob sie für die Produkte des Kindes bezahlen wollen.


\section{Zusätzliche Services}


\subsection{Weitere Entwicklungen auf der Basis von Ebay Kleinanzeigen}

Von Ebay Kleinanzeigen zum online Flohmarkt.

Gebrauchte Produkte werden auf der Webseite angezeigt. Die Produkte werden nicht mehr nach einzelnen Attributen klassifiziert, sondern werden gemäß ihrem Verkäufer gruppiert. Ähnlich wie bei dem Besuch eines Flohmarktes. Die gebrauchten Produkte werden nicht nach Attributen sortiert aufgestellt. Z.B. werden nicht alle Spielzeuge des Flohmarkts an einer Stelle verkauft, sonder die Produkte sind nach ihrer Quelle, ihrem Verkäufer sortiert.


\subsection{Communities  für die Nutzer erstellen, um Nachrichten zu veröffentlichen und Erfahrung auszutauschen}

\begin{itemize}
\item Kooperation existiert vor allem in Computerspielen
\item Darstellung und Interaktion im Webshop wird denen von Computerspielen nachempfunden
\item 3D Darstellung und Steuerung wie in Computerspielen
\end{itemize}


\section{Nutzer-Community}

Nutzer- Community,  die Einkaufskommunikationsgemeinschaft des Käufers, dient die Tipps oder Feedback vor und nach dem Kauf zu teilen.Ob eine Ware teuer ist oder nicht, Menschen werden vor dem Kauf eine Untersuchung machen, Freunde fragen, google oder  Bewertung lesen.  Community ist ein Ort, um die Informationsangreifen und Kommunikation vor dem Kauf vorzunehmen. Zweitens wird es nach dem Kauf der Waren soziale Bedürfnisse geben. Manche Leute wollen die gekaufte Produkte zeigen oder kritisieren , solche Informationen sind für andere Nutzer wertvoll.  immer mehr Menschen kommunizieren gerne in der Gemeinschaft.

Üblich bewertet man die Nutzer-Community von Web-Shops , welche nötige Hilfe oder Tipps verfügbar in der Seite sind. Das laut Meinung der Autorin primäre und wichtigste Kriterium ist hierbei, die möglichkeit von Tipps und Feedback zu geben. Online Hilfe und Richtlinien spielen laut Meinung der Autorin eine untergeordnete Rolle. Tabelle \vref{tab:tabelle-gruen} zeigt die Gegenüberstellung der ausgewählten Webshops.

\begin{table}[htbp]
	\centering
	\includegraphics[width=1\textwidth]{bilder/tabelle-gruen.png}
	\caption{Nutzer-Community}
	\label{tab:tabelle-gruen}
\end{table}

In der Hinsicht bieten Alibaba und Bandcamp eine gute Plattform, sodass Nutzer miteinander Kommunikation machen können. Bei Alibaba geschieht dies durch ein separates Forum\footnote{Alibaba Community: \url{https://buyer.alibaba.com/forum}}. Bei Bandcamp\footnote{ Bandcamp Community: \url{https://bandcamp.com}} kann man nicht nur mit den Künstler verbinden, sondern auch mit anderen Fans die Erfahrung verteilen.

Obwohl Amazon\footnote{Amazon Community: \url{https://www.amazon.de/gp/help/customer/display.html?nodeId=201483620}} und Ebay\footnote{Ebay Community: \url{https://pages.ebay.de/help/policies/member-created-content-ov.html\#policy}} bieten auch die Community an. Deren Funktionen orientieren sich an der online- Hilfe und Richtlinien, keine Praktische Benutzerinteraktion verfügbar. Ein Ausnahme ist Zalando, die Firma liegt die Community auf Facebook\footnote{Zalando Community: \url{https://de-de.facebook.com/pg/Zalando.co.uk/community/}}.

\begin{figure}[htbp]
	\centering
	\includegraphics[width=1\textwidth]{bilder/zalando-community.png}
	\caption{Zalando Community auf Facebook}
	\label{fig:zalando-community}
\end{figure}

Wie in Abbildung \vref{fig:zalando-community} zu sehen, hat diese Maßnahme auch einige Nachteile. Z. B. muss mit Facebook koexistieren, kann nicht die Kommentare kontrollieren, viele Nutzer wissen nicht, wo ist die Community von Zalando usw.


\section{Online Kommunikation}

Um die online Kommunikation von Web-Shops zu bewerten, wurden durch die Autorin die Kriterien Real-Time Chatten, Fotos senden und verfügbare Sprachkanäle ausgewählt. Das laut Meinung der Autorin primäre und wichtigste Kriterium ist hierbei, die Möglichkeit zum Real-Time Chatten. Fotos senden und zusätzliche Sprachkanäle spielen laut Meinung der Autorin eine untergeordnete Rolle.

Online Kommunikation (Realtime – chatten) spielt für die Onlineshops immer größer Rolle. Die Statistik von \textcite{aliyun}\footnote{Tochterfirma der Alibaba} zeigt, bis 12.2016 verwenden 100 Millionen Nutzer Aliwangwang. Eine spezielle Software für online chatten, Hilfswerkzeug von Alibaba. Amazon unterstützt offensichtlich nicht diese Funktion.

Tabelle \vref{tab:tabelle-grau} stellt die Online Kommunikation der ausgewählten Webshops gegenüber.

\begin{table}[htbp]
	\centering
	\includegraphics[width=1\textwidth]{bilder/tabelle-grau.png}
	\caption{Online Kommunikation}
	\label{tab:tabelle-grau}
\end{table}

Die Autorin hat zufällig ein Produkt ausgewählt. Von diesem Bild zeigt es, der Nutzer am 15.04.2018 eine Frage erstellt, aber antwortete der Verkäufer am 18.04. 2018 diese Frage. Das folgt möglich zu, dieser Handelsprozess kann nicht gut laufen, weil die Kommunikation blockiert werden.

Zalando hat ähnliche Problem wie Amazon, auf ihre Webseite steht keine Online- Kommunikation. \parencite{piatscheck}


\section{Produktvielfalt}

Jedes Web-shop orientiert sich bei der Gründung bestimme Kundengruppe.  Am meisten verwendet werden die B2B , C2C und B2C Beziehungen. B2B steht für Business-to-Business und bezeichnet eine Handelsbeziehung, wenn Käufer und Verkäufer Unternehmen sind. B2C bedeutet Business-to-Consumer und bezieht sich auf Transaktionen zwischen Unternehmen (Anbieter) und Endkunden (Käufer). C2C bezeichnet ein Handel zwischen Kunden und Kunden.

Die Vielfalt der Produkte hat starken Einfluss auf den Umsatz. Nach einer Statistik von \textcite{statista} liegt Amazon auf dem 1. Platz und Zalando liegt auf den 3. Platz. Ebay, Bandcamp und Alibaba sind B2B – Form von E-Commerce.

Nach obiger Statistik von \textcite{statista} wird es bestätigt, der Gewinn ist abhängig von den vielfalten Produkten. Jetzt können wir  untersuchen, bei welchem Businessmodell werden die Firmen vielfalte Produkte verkaufen. Nach die Meinung von Richard Lazazzera, How to Choose an Ecommerce Business Model\footnote{\url{https://www.shopify.com/blog/17240328-how-to-choose-an-ecommerce-business-model}} wird Businessmodell unterschiedliche Kriterium unterteilt. Davon ist die Beziehung zwischen Käufer und Verkäufer wichtigst, dann können Marktpositionierung und Speziell Bereich den zweiten Platzen liegen.

Tabelle \vref{tab:tabelle-rot} stellt die Produktvielfalt der ausgewählten Webshops gegenüber.

\begin{table}[htbp]
	\centering
	\includegraphics[width=1\textwidth]{bilder/tabelle-rot.png}
	\caption{Produktvielfalt}
	\label{tab:tabelle-rot}
\end{table}


\section{Zahlungsmethoden}

Weil Alibaba eigne Pay- system besitzt, Alipay( www. Alipay.com), kann die Handelsbereich nach Finanz erweitern. Unter die Bedingung von Sicherheit kann Nutzer mehrere Möglichkeit zu handeln  machen. Aber andere Online Shops verwenden noch die Finanzprodukt von anderen Unternehmen, z.B. Bank, paypal usw.

Um die Zahlungsmethode von Web-Shops zu bewerten, wurden durch die Autorin die Kriterien Kreditkarte, Giro Karte, online Bezahldienste und eigne Finanzservice ausgewählt. Das laut Meinung der Autorin primäre und wichtigste Kriterium ist hierbei, die Möglichkeit zum Pay bei Others. Bei eigner Finanzservice kann man die Finanzprodukt riesig skalierbar, deswegen spielen laut Meinung der Autorin auch eine große Rolle.

Tabelle \vref{tab:tabelle-rot} stellt die Zahlungsmethoden der ausgewählten Webshops gegenüber.

\begin{table}[htbp]
	\centering
	\includegraphics[width=1\textwidth]{bilder/tabelle-blau.png}
	\caption{Zahlungsmethoden}
	\label{tab:tabelle-blau}
\end{table}


\section{Methoden der Bewertung}

Bei der Bewertung bieten Amazon und Alibaba unterschiedliche Vorgehensweise zu kommentieren,z.B. die Bewertung ist bei Amazon in Form Stern Plus Text. 1 – 5 Stern zeigen die Level der Unzufrieden bis sehr zufrieden. Abbildung \vref{fig:bewertung-amazon} zeigt Bewertungen in Amazon.

\begin{figure}[htbp]
	\centering
	\includegraphics[width=1\textwidth]{bilder/bewertung-amazon.png}
	\caption{Bewertungen in Amazon}
	\label{fig:bewertung-amazon}
\end{figure}

Nach die Kommentare unter dem Produkt zeigt es, immer mehrere Kunden wollen ihre Bewertung geben, damit andere Nutzer Tipps oder Hinweise erhalten können. Die online- shops erstellen unterschiedliche Methode, sodass die Bewertung konkreter und intuitiver sind. Nach die Meinung der Autorin sind die Methode für die Nutzer wichtiger und nötiger.

Tabelle \vref{tab:tabelle-gelb} stellt die Methoden der Bewertung für die ausgewählten Webshops gegenüber.

\begin{table}[htbp]
	\centering
	\includegraphics[width=1\textwidth]{bilder/tabelle-gelb.png}
	\caption{Methoden der Bewertung}
	\label{tab:tabelle-gelb}
\end{table}

\newpage


\section{Gesamtergebnis}

Tabelle \vref{tab:tabelle-ergebnis} stellt das Gesamtergebnis der ausgewählten Webshops gegenüber.

\begin{table}[htbp]
	\centering
	\includegraphics[width=1\textwidth]{bilder/tabelle-ergebnis.png}
	\caption{Gesamtergebnis}
	\label{tab:tabelle-ergebnis}
\end{table}

Nach der Rechnung der Gesamtpunkte bei unterschiedlichen Firmen können wir durch ein Diagramm zeigen, welche Firma ist dominant. Abbildung \vref{fig:diagramm-ergebnis} stellt das Gesamtergebnis dar.

\begin{figure}[htbp]
	\centering
	\includegraphics[width=0.7\textwidth]{bilder/diagramm-ergebnis.png}
	\caption{Auswertung Gesamtergebnis}
	\label{fig:diagramm-ergebnis}
\end{figure}

Durch das Diagramm zeigt das Ergebnis, vergleich zu anderen Firmen hat Alibaba höchste Punkte. Unser online – shops kann mittels das Vorteil von Alibaba die Service- Qualität bei der  online- Geschäft erhöhen.