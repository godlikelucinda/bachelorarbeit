% !TEX root =../main.tex

\documentclass[%
	a4paper,%				A4 Papier
	oneside,%				einseitig
	bibliography=totoc,%		Literaturverzeichnis einfügen
	%bibtotocnumbered,% 		Literaturverzeichnis einfügen nummeriert
	listof=totoc,%			Verzeichnisse einbinden in toc
	index=totoc,%			Index ins Verzeichnis einfügen
	parskip=half,%			europäischer Satz mit abstand zwischen Absätzen
	%nochapterprefix,%		keine Ausgabe von 'Kapitel:'
	chapterprefix,%			Ausgabe von Kapitel und Anhang
	headsepline,%			Linie nach Kopfzeile
	%footsepline,%			Linie vor Fußzeile
	12pt,%					grössere Schrift, besser lesbar am Bildschirm
]{scrbook}

%
% Unterstützung für deutsche Sprache
%
\usepackage[ngerman]{babel}
\usepackage[utf8]{inputenc}
\usepackage[T1]{fontenc}
\usepackage[babel,german=quotes]{csquotes}

%
% Schriftart
%
%\usepackage{bera}
%\usepackage{times}
\usepackage{palatino}

%
% Pakete Literarische Quellen und Literaturverzeichnis
%
\usepackage[backend=biber]{biblatex}
\bibliography{literatur}

%
% Paket um Grafiken einbetten zu können
%
\usepackage{graphicx}

%
% Paket für Querverweise
%
\usepackage[ngerman]{varioref}

% 
% Pakete für Tabellen
%
\usepackage{longtable}
\usepackage{tabularx}
\usepackage{booktabs}
% Damit kann man an Stelle des p{BREITE} einfach L{Breite} für linksbündige,
% C{BREITE} für zentrierte, und R{BREITE} für rechtsbündige Textausrichtung
% verwenden.
\newcolumntype{L}[1]{>{\raggedright\arraybackslash}p{#1}} % linksbündig mit Breitenangabe
\newcolumntype{C}[1]{>{\centering\arraybackslash}p{#1}} % zentriert mit Breitenangabe
\newcolumntype{R}[1]{>{\raggedleft\arraybackslash}p{#1}} % rechtsbündig mit Breitenangabe
% Um innerhalb einer einzelnen Spalte eine andere Textausrichtung als die
% vordefinierte nutzen zu können, werden folgende Befehle definiert.
% Jetzt kann man innerhalb der gewünschten Spalte folgende Befehle vor den Text schreiben:
%    \ctab = zentrierte Ausrichtung
%    \rtab = rechtsbündige Ausrichtung
%    \ltab = linksbündige Ausrichtung
\newcommand{\ltab}{\raggedright\arraybackslash} % Tabellenabschnitt linksbündig
\newcommand{\ctab}{\centering\arraybackslash} % Tabellenabschnitt zentriert
\newcommand{\rtab}{\raggedleft\arraybackslash} % Tabellenabschnitt rechtsbündig

%
% Paket für Links innerhalb des PDF-Dokuments
% (hyperref muss unbedingt vor harvard geladen werden)
%
\usepackage[pdfborder={0 0 0}]{hyperref}

%
% Pakete für spezielle Formatierungsprobleme
%
\usepackage{xspace}

%
% Selbst definierte Befehle
%
\newcommand{\code}[1]{\texttt{#1}}
\newcommand{\latex}{\LaTeX\xspace}
\newcommand{\tex}{\TeX\xspace}